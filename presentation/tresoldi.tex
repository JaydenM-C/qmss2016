% Tiago Tresoldi presentation

\documentclass{beamer}

\usepackage{graphicx}

\usepackage{./beamerthemempiis}

\usepackage{colortbl}

\usepackage{tipa}

\usepackage{hyperref}

\usepackage{amssymb}
\let\oldemptyset\emptyset
\let\emptyset\varnothing

\begin{document}
\title{QMSS16 Presentation: \\ Distinct Phonetic Features for Language Phylogeny}   
\author{Tiago Tresoldi} 
\date{May 18\textsuperscript{th}, 2016} 


\frame{\titlepage} 


\section{The questions and the problems} 
\frame{\frametitle{The questions and the problems}
\begin{itemize}
\item Is it possible (and useful) to integrate phonology in language phylogeny?
	\begin{itemize}
	\item Phonology has a high rate of evolution, but borrowing is rare
	\item Diachronic and synchronic data allow to estimate the probability of future changes
	\end{itemize}
\item Hypothesis: using distinct features (``the most basic unit of analysis in
      phonetic structures''), adapting Roman Jakobson \textit{et al.} (1941--1956),
	  Chomsky and Halle (1968--1983), and, in particular, Ladefoged (1971; 1993)
	\begin{itemize}
	\item Pros: binary (and mostly non-exclusive) features (no dummy variables!),
	      single description for vowel and consonants
	\end{itemize}
\item Traditional reconstruction (such as for the IE family) uses words mostly as proxy for phonology,
      but we should combine lexical and phonological data
\end{itemize} 
}



\frame{\frametitle{The data}

\begin{table}[]
\centering
\caption{Phonetic reflexes (snippet)}
\label{table-phonemes}
\begin{tabular}{lllll}
PIE & Old Irish & Latin & Old English & Lithuanian \\
*p   & $\emptyset$         & p     & f           & p          \\
*\'{g}\textsuperscript{h}i & g         & h     & g           & \textipa{Z}          \\
*k\textsuperscript{w}  & k         & k\textsuperscript{w}    & h\textsuperscript{w}          & k         
\end{tabular}
\end{table}

\begin{table}[]
\centering
\caption{Distinct features of phonetic reflexes (snippet)}
\label{table-reflexes}
\begin{tabular}{llllllll}
            & p.ant & p.obstr & p.lab & \'{g}\textsuperscript{h}i.obstr & \'{g}\textsuperscript{h}i.cor \\
Old Irish   & -     & -       & -     & T       & F           \\
Latin       & F     & T       & T     & T       & F           \\
Old English & T     & T       & F     & T       & F           \\
Lithuanian  & F     & T       & T     & T       & T          
\end{tabular}
\end{table}

}

\section{The (very, very first) results}
\frame{\frametitle{The (very, very first) results}
\includegraphics[scale=0.41]{tree.png}
}

\section{Conclusion} 
\frame{\frametitle{Conclusion}
\begin{itemize}
\item Without constrains and additional data, the tree is
      mirroring similarities in phoneme inventories, not language history

\item Ancestral State Recostruction -- do results from past methods match those of current ones?
\begin{itemize}
	\item Data from MPI banks (PHOIBLE, ``Intercontinental Dictionary Series'', 
	  ``World Loanword Database''), and paper dictionaries
	\item Tools: phylogenetic software, lingpy
	\item Combine known data and phylogeny to infer best model and order of ancestral state changes
\end{itemize}

\item Code on GitHub (\url{https://github.com/tresoldi/qmss2016})

\item Dreamy questions:
	\begin{itemize}
	\item How confident can we be on the presence of laryngeals in PIE? What is the most likely
	      date for their evolution? How many there were?
	\item How confident can we be of Etruscan as a colonial Luwian language? (cf. Woudhuizen, 2008)
	\end{itemize}
\end{itemize} 
}

\end{document}